\subsection{CalculatorView类}
本小节依照用户信息的输入顺序阐述消息传递的具体实现。在设计之初,为CalculatorView添加
CString成员变量m\_input用于储存用户在编辑框输入的字符串。
现在假设用户要计算$1+2$,
当用户点击按钮"1",系统会调用CalculatorView中其对应的消息响应函数:
\begin{lstlisting}
    void CCalculatorView::OnBnClickedButton1()
    {
        m_input += _T("1");
        UpdateData(FALSE);
    }
\end{lstlisting}
此函数将字符“1”添加在m\_input末尾,并将m\_input的值在窗口更新。"+"和"2",以及其他所有
数字和运算符的函数都同理。

在这之后,用户会点击“=”,对应的消息响应函数如下:
\begin{lstlisting}
    void CCalculatorView::OnBnClickedButtonEqual()
    {
	if (m_FractionMode.GetCheck() == BST_UNCHECKED) {
		evaluate(m_input, 1);
	}
	else {
		evaluate(m_input, 2);
	}
	UpdateData(FALSE);
    }
\end{lstlisting}
m\_FractionMode也是CalculatorView类的成员变量,其关联对话框中的“分数模式”
勾选状态,通过检测其状态来确定evaluate函数的模式,evaluate负责计算结果并更新m\_input。
至此,完成了$1+2$的计算。

在点击菜单中“表述性统计”时,调用下面的函数:
\begin{lstlisting}
    void CCalculatorView::OnStatDescribe()
    {
	    CDescribeStat ds;
	    ds.DoModal();
    }
\end{lstlisting}
此函数调用CDescribeStat类的DoModal函数,生成新的对话框,“矩阵运算”
同理。

为了实现“高级”中设置控件可见性的功能,引用了辅助函数
checkVisibility:
\begin{lstlisting}
    void CCalculatorView::checkVisibility(int ID)
    {
        // 检查控件当前是否可见
        CWnd* pControl = GetDlgItem(ID);
        if (pControl && pControl->IsWindowVisible())
        {
            pControl->ShowWindow(SW_HIDE);  // 当前可见则隐藏
        }
        else
        {
            pControl->ShowWindow(SW_SHOW);  // 当前隐藏则显示
        }
    }
\end{lstlisting}
利用此函数,可以轻易实现“高级”和“分数模式”的消息处理函数。