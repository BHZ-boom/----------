\section{困难及解决方案}
本节将介绍开发过程中遇到的困难,以及解决方法。

\subsection{框架设计}
在开发之初,由于对MFC不够熟悉,我们遇到了许多奇怪的错误。
我们想找一个既有主菜单,又以对话框为主界面的框架,找寻好久
才找到基于单文档的继承CFormView类的MFC框架;
不知道要把预编译头文件"pch.h"放在头文件首位,
导致自己定义的类总是构建失败;因为不熟悉各个类具体的功能,
不知道应该在哪里添加消息处理函数。这些问题最终都通过查阅资料解决。

\subsection{消息处理}

\subsubsection*{特殊字符输入}

在编辑对话框的时候,首先遇到的问题就是如何输入并读取
除号、派等特殊字符,对此,我们查询了这些字符的Unicode
编码,直接使用编码来显示字符,如除号在代码中就是L'\textbackslash u00F7'。

\subsubsection*{纠错机制}

在测试过程中,会出现不小心连续输入两个运算符的情况,
比如“1++2”,这样计算程序会直接崩溃,为了增加程序健壮性,
我们增加了纠错机制,每次输入字符,程序都会检查是否出现连续的
运算符,如果出现,则忽略输入或为其添加括号,具体程序如下:
\begin{lstlisting}
    void CCalculatorView::OnEnChangeEdit()
    {
        int length = m_input.GetLength();
        wchar_t last = m_input[length - 1], 
            second = m_input[length - 2];
        // 如果倒数第二个符号不是“-”
        if (second == L'+' || second == L'\u00D7' || second == L'\u00F7') {
            // 置换为倒数第一个字符
            if (last == L'+' || last == L'\u00D7' || last == L'\u00F7') {
                m_input.Delete(length - 2);
            } 
            // 如果倒数第一个字符是“-”,当成符号,为其加括号
            else if (last == L'-') {
                m_input.Insert(length - 1, L'(');
            }	
        }
        // 如果倒数第二个符号是“-”,则置换为倒数第一个字符
        else if (second == L'-') {
            if (last == L'-' || last == L'+' || last == L'\u00D7' || last == L'\u00F7') {
                m_input.Delete(length - 2);
            }
        }
        UpdateData(FALSE);
    }
\end{lstlisting}

\subsection{字符串处理与运算}
起初,我们的分数类使用int类型作为分子和分母;使用double类型读取小数。
但实际测试中,我们发现运算很容易溢出(超过十亿就会溢出)。为了解决这个问题,
改用long long类型作为分子和分母;使用long double类型读取小数,实测改进后
支持$10^{18}$以内的运算,覆盖了绝大多数使用范围。

在数量级较大时,人眼不易确定具体的位数,所以每三位数使用逗号分割(分数模式不适用)。

在读取字符串的时候还发现了一个错误,原先对于字符串中数字是这样读取的:
\begin{lstlisting}
    BOOL ifDecimal = FALSE;
    while (i < expression.GetLength() && (isdigit(expression[i]) || expression[i] == L'.' || expression[i] == L',')  ) {
        ......
        val = (val * 10) + (expression[i] - '0');
        if (ifDecimal) val /= 10;
        i++;
    }
\end{lstlisting}
但这样实际上无法正确处理小数部分,经过调试,修正为:
\begin{lstlisting}
    BOOL ifDecimal = FALSE;
    long  decimalDigit = 1; //小数位数
    while (i < expression.GetLength() && (isdigit(expression[i]) || expression[i] == L'.' || expression[i] == L',')  ) {
        ......
        val = (val * 10) + (expression[i] - '0');
        if (ifDecimal) decimalDigit *= 10;
        i++;
    }
    val /= decimalDigit;
\end{lstlisting}

\subsection{描述性统计}
在使用ControlList的过程中,遇到了奇怪的现象,即使文本超出窗口,也不会出现水平滚动条。
查阅资料无果,经过大量测试后,发现只有初始状态有竖直滚动条,才能出现水平滚动条。
在进行数据保存时,我们遇到了编码问题,VS2022中MFC采用UTF-16进行编码,而Windows自带的
txt文本查看器采用UTF-8编码,直接保存会呈现乱码,查阅资料后我们加入了转码程序,使得数据正常保存。


\subsection{矩阵运算}
实现读取矩阵的程序时,我们忽略了windows中换行的独特表示,一开始我们用'\textbackslash n'表示换行符,
但没有作用,查阅资料后改为'\textbackslash r \textbackslash n',成功实现矩阵的读取。










