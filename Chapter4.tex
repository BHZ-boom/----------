\section{总结}
\subsection{不足}
\begin{itemize}
    \item 没有构建严格的输入纠错程序,如果输入的算式不符合数学格式,程序可能崩溃。
    \item 由于矩阵逆运算的复杂性,只提供了加、减、乘三种运算。
    \item 没能允许用户自主选择数据库中要统计的列,而是直接统计整个表格。
    \item 字符串处理程序没有很好地实现模块化,显得臃肿难以维护。
\end{itemize}

\subsection{收获}
\begin{itemize}
    \item 本次开发协作使用Git及Github完成,虽然过程中出现了登录失败等问题,
    但最终有效提高了效率,省去了使用队友代码时麻烦的配置工作,锻炼了合作开发的能力。
    \item 开发过程中查资料、调试,以及对代码整体框架的设计锻炼了工程能力。
    \item 增加了多文件程序以及窗口化程序的开发经验。
\end{itemize}
这次程序设计让我们意识到理论与实际之间地鸿沟,计算机、字符串处理地程序逻辑都不难理解,
可是要真正实现所有功能,解决各种奇奇怪怪的bug,提供可视化操作界面,需要数千行代码的支撑。
同时,这次作业也让我们熟练了C++类的用法,以及明白可视化程序的设计思路。理论指导实践,这次课程设计
是一次充实有趣的实践,让我们收益匪浅!






