\subsection{Fraction类}
Fraction类负责提供分数数据类型,
支持分数的存储,四则运算,以及基本函数运算。
\begin{lstlisting}
    class Fraction
    {
    private:
        long long m_numerator;
        long long m_denominator;
    public:
        //构造函数
        explicit Fraction(long long above = 0, long long below = 1); 
        Fraction(const Fraction& rhs); //复制构造函数
        Fraction(long double input); //小数转化成分数
        void reset(long long above, long long below); //重置分数值
        void gcd(); //化简
        long long up(); //返回分子的值
        long long down(); //返回分母的值
        friend Fraction operator+(const Fraction& left, 
            const Fraction& right);
        ...... // 四则运算符重载
        friend Fraction pow(Fraction& left, 
            Fraction& right);
        .... // 幂函数、对数函数等函数的重载
    };    
\end{lstlisting}
为了尽可能提高运算精度,分子和分母均为long long类型。大多数成员函数的
实现都比较简单,具体可查看附录,此处具体解释小数转分数的构造函数实现:
\begin{lstlisting}
    Fraction::Fraction(long double decimal) {
        // 取传入小数的整数部分
        long long intPart = static_cast<long long>(decimal);
        if (intPart == decimal) {
            // 整数部分等于整体,说明小数部分为0,传入的是整数
            m_denominator = 1;
            m_numerator = intPart;
        }
        else {
            const long precision = 1000000; // 定义精度
            // 将传入小数乘以精度后取整,即得到分子
            this->m_numerator = static_cast<long long>(round(decimal * precision));
            this->m_denominator = precision;
    
        }
        // 化简以精度为分母的分数
        this->gcd();
    }
\end{lstlisting}
为了使转化后的分数视觉上直观,设置了转化精度。


