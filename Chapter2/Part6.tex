\subsection{CMatrixManip类}
CDescribeStat类是\autoref{fig:MatrixManip}
对话框的关联类,添加的成员函数如下:
\begin{lstlisting}
public:
    //定义矩阵数据类型
    typedef std::vector<std::vector<long double>> Matrix; 
    // 存储编辑框中输入的内容
	CString m_A;
	CString m_B;
	CString m_C;

    // 将m_A,m_B,内容读取为矩阵类型m_MA,m_MB
    Matrix m_MA, m_MB;
    void readOneMatrix(const CString& input, Matrix& matrix);
    void readMatrix(); 

    // 矩阵加法、减法、乘法的消息处理及运算
	afx_msg void OnBnClickedButtonPlus();
	afx_msg void OnBnClickedButtonMinus();
	afx_msg void OnBnClickedButtonMul();
	
    // 将矩阵类型转化为字符串
	void Matrix2String(const Matrix& a, CString& b);
\end{lstlisting}


readOneMatrix函数使用CString自带的Tokenize方法,使用循环将字符串按矩阵的行分割,
然后将每一行分割为多个元素,储存到自定义的Matrix类型中。
\begin{lstlisting}
    void CMatrixManip::readOneMatrix(const CString& input,Matrix& matrix)
    {
        int pos = 0;
        CString token;
        CString delimiters = _T("\r\n"); // 列分隔符
    
        // 分割字符串为多个行
        CString rowString = input.Tokenize(delimiters, pos);
        while (pos != -1) {
            std::vector<long double> row;
            int rowPos = 0;
            CString rowDelimiter = _T(" "); // 行内元素分隔符
            CString element = rowString.Tokenize(rowDelimiter, rowPos);
    
            // 分割行为多个元素
            while (rowPos != -1) {
                // 转换CString到long double
                long double num = _tcstold(element, NULL); 
                row.push_back(num);
                element = rowString.Tokenize(rowDelimiter, rowPos);
            }
    
            matrix.push_back(row);
            // 当剩余字符串没有分隔符,返回-1
            rowString = input.Tokenize(delimiters, pos); 
        }
    }
\end{lstlisting}
其余函数均容易实现,具体代码及注释见附录CMatrixManip.cpp。


